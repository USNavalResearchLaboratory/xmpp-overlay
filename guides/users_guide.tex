\documentclass[draft]{article}

\usepackage{hyperref}
\usepackage{todonotes}
\usepackage{placeins}
\title{XO Users' Guide}
\date{\today}
\author{Trevor Adams, Robert Lass, and David Millar}

\begin{document}
\maketitle


\section{Introduction}
XOP (Xmpp Overlay Proxy) is a proxy for XMPP clients that provides a distributed
messaging environment for MANETs.  It is essentially a light-weight
XMPP server that supports components to extend its functionality.  XMUM for example
is a plugin that provides distributed group chat for users of a network that has
been tailored for MANET environments.


\section{Obtaining XOP}
There are two ways to obtain XOP: one is to build it from source the other is to download a binary distribution.
\subsection{Binary Distribution}

Binary distributions can be obtained from: \todo{Location here}

\subsection{Source}
XOP is hosted at the Naval Research Lab's (NRL) proteanforge repository, located at
\url{https://pf.itd.nrl.navy.mil/svnroot/groupwise/xop/trunk}. In order to obtain
a copy of the XOP source, you will need to check it out with an svn client as in
Figure~\ref{fig:svn-export-output}.

\begin{figure}
\begin{verbatim}
$svn export https://pf.itd.nrl.navy.mil/svnroot/groupwise/xop/trunk
A    trunk
A    trunk/test
A    trunk/test/TestSample.java
A    trunk/test/edu
A    trunk/test/edu/drexel
A    trunk/test/edu/drexel/xop
A    trunk/test/edu/drexel/xop/core
A    trunk/test/edu/drexel/xop/core/TestClientProxy.java
A    trunk/test/edu/drexel/xop/core/TestJidPacketFilter.java
...
A    trunk/docs
A    trunk/docs/CHANGELOG.TXT
A    trunk/xmpp_samples
A    trunk/xmpp_samples/example_stanzas
A    trunk/xmpp_samples/example_stanzas/Spark Joins Openfire Muc.xml
A    trunk/xmpp_samples/example_stanzas/XMPP MUC Operations Condensed.xml
A    trunk/xmpp_samples/example_stanzas/openfire spark login.xml.xml
A    trunk/xmpp_samples/example_stanzas/XMPP MUC Operations.xml
A    trunk/build.xml
Exported revision 329.
\end{verbatim}
\caption{Example output from exporting XOP}
\label{fig:svn-export-output}
\end{figure}

This will check out the latest trunk version of the XOP source and related
artifacts.


\subsubsection{Creating a Build}
XOP comes bundled with an ant config file that specifies a number of targets.
In order to build a XOP distribution, simply run \textbf{ant dist} as seen
in Figure~\ref{fig:ant-build}. After this, a dist folder will be created
that contains the XOP distribution.

\begin{figure}
\begin{verbatim}
$ant dist
Buildfile: C:\Users\Dave\code\xop\trunk\build.xml

init:
    [mkdir] Created dir: C:\Users\Dave\code\xop\trunk\build
    [mkdir] Created dir: C:\Users\Dave\code\xop\trunk\dist

compile:
    [javac] Compiling 80 source files to C:\Users\Dave\code\xop\trunk\build

javadoc:
  [javadoc] Generating Javadoc
  [javadoc] Javadoc execution
  [javadoc] Creating destination directory: "C:\Users\Dave\code\xop\trunk\docs\javadoc\"
  [javadoc] Loading source files for package edu.drexel.xop...
  ...

jar:
      [jar] Building jar: C:\Users\Dave\code\xop\trunk\dist\xop.jar

test-compile:
    [javac] Compiling 12 source files

test:
     [echo] ${test.classpath}
    [junit] Running edu.drexel.xop.util.TestThreadedPacketDeliveryQueuePacketListener
    [junit] Testsuite: edu.drexel.xop.util.TestThreadedPacketDeliveryQueuePacketListener
    [junit] Tests run: 1, Failures: 0, Errors: 0, Time elapsed: 0 sec
    [junit] Tests run: 1, Failures: 0, Errors: 0, Time elapsed: 0 sec
    ...

guides:
     [move] Moving 2 files to C:\Users\Dave\code\xop\trunk\docs

dist:
     [copy] Copying 15 files to C:\User\Dave\code\xop\runk\dist\lib
     [copy] Copying 2 files to C:\Users\Dave\code\xop\trunk\dist\config
     [copy] Copying 1 file to C:\Users\Dave\code\xop\trunk\dist\plugins
     [copy] Copying 137 files to C:\Users\Dave\code\xop\trunk\dist\docs

BUILD SUCCESSFUL
Total time: 6 seconds
\end{verbatim}
\caption{Example ant build}
\label{fig:ant-build}
\end{figure}

This creates a new directory called dist in the current directory.
This directory contains the executable code, configuration files,
and documentation for your XOP distribution.

To clean up a XOP distribution, run \textbf{ant clean} as in Figure~\ref{fig:ant-clean}

\begin{figure}
\begin{verbatim}
$ant clean
Buildfile: C:\Users\Dave\code\xop\trunk\build.xml

clean-test:

clean:
   [delete] Deleting directory C:\Users\Dave\code\xop\trunk\build
   [delete] Deleting directory C:\Users\Dave\code\xop\trunk\dist
   [delete] Deleting directory C:\Users\Dave\code\xop\trunk\docs\javadoc

BUILD SUCCESSFUL
Total time: 0 seconds
\end{verbatim}
\caption{Cleaning up a build}
\label{fig:ant-clean}
\end{figure}

This will remove everything that was created by running \textbf{ant dist},
effectively restoring things to the way they were when XOP was exported.

\FloatBarrier
\section{Configuration}
There are two primary ways that a user can configure XOP.  One is by modifying
the xop properties file.  The other way is to add and remove components.

\subsection{XOP Properties}
Once a build is created, a user can modify the XOP properties file to change its
behavior.  The properties file is located in \texttt{dist/config/xopProps.xml}. 
Note if you wish for the configuration to persist over builds, edit \texttt{config/xopProps.xml} instead.
It is a set of name-value pairs that determine how a particular behavior of XOP
works.

An example properties file is shown in Figure~\ref{fig:sample-config-file}.
\subsubsection{\texttt{xop.*}}
\paragraph{\texttt{xop.debug}}
Client connection debugging, logs the client packet information to \texttt{/tmp/tcp-\$UUID}

Default: true
\paragraph{\texttt{xop.domain}}
Which domain should XOP use for its base (eg: bob@DOMAIN)

Default: proxy
\paragraph{\texttt{xop.device}}
Which device (eth0/eth1) should XOP use for its network traffic.

Default: undef
\paragraph{\texttt{xop.s2s.whitelist}}
Server to Server whitelist, comma delimited. 

Default: undef
\subsubsection{\texttt{xop.client}}
\texttt{xop.client} properties define how the client portion of XOP should behave.
\paragraph{\texttt{xop.client.authentication.provider}}
What authentication provider should XOP use, options: 
\begin{enumerate}
    \item \texttt{SimpleAuthenticationProvider} -  Accept all usernames and passwords (no checking)
\end{enumerate}

Default: \texttt{edu.drexel.xop.client.SimpleAuthenticationProvider}
\paragraph{\texttt{xop.client.port}}
Which port should XOP accept client connections on. 

Default: 5222
\subsubsection{\texttt{xop.router}}
\paragraph{\texttt{xop.router.core.pool.size}}
Configure the core thread pool size for the router.

Default: 0 
\paragraph{\texttt{xop.router.max.pool.size}}
Configure the maximum thread pool size for the router.

Default: MAX INT
\paragraph{\texttt{xop.router.iq.timeout}}
How long to wait for a component to respond to an IQ request (an error is returned if there is no response in this time period)

Unit: milliseconds

Default: 2000
\subsubsection{\texttt{xop.component}}
Configuration section for component loading.
\paragraph{\texttt{xop.component.autoload}}
Should XOP load components, true/false.

Default: true
\paragraph{\texttt{xop.component.autoload.poll.period.ms}}
How often should XOP check the autoload directory.

Default: 5000
\paragraph{\texttt{xop.component.autoload.dir}}
Which directory should XOP look for its plugins.

Default: \texttt{./plugins/}
\subsubsection{\texttt{Gump.*}}
Options with the prefix \texttt{Gump.} are passed directly to GUMP.
\paragraph{\texttt{Gump.Discovery}}
Which implementation should be used for discovery.

Default: \texttt{mil.navy.nrl.discovery.imp.jmdns.JmDNSInstance}
\paragraph{\texttt{Gump.Multicast}}
Which multicast implementation should be used for discovery, options:

\begin{enumerate}
    \item \texttt{JDKMulticastSocket} - UDP Multicast
    \item \texttt{NormMulticast} - NRL NORM Multicast (configurable via Gump.Norm.*)
\end{enumerate}

Default: \texttt{mil.navy.nrl.protean.gump.transport.multicast.imp.JDKMulticastSocket}
\paragraph{\texttt{Gump.Transport}}
Not used by XOP.

Default: \texttt{mil.navy.nrl.protean.gump.transport.input.imp.tcp.TCPInstance}
\paragraph{\texttt{Gump.Proxy}}
Not used by XOP.

Default: \texttt{mil.navy.nrl.protean.gump.proxy.imp.NullProxy}
\begin{figure}
\begin{verbatim}
#XOP Properties File
xop.device=eth0
xop.debug=true
xop.domain=proxy
xop.s2s.whitelist=fireopen,conference.fireopen

xop.client.authentication.provider=edu.drexel.xop.client.SimpleAuthenticationProvider
xop.client.connection.reap.period.ms=5000
xop.client.port=5222

xop.router.core.pool.size=10
xop.router.max.pool.size=20
xop.router.iq.timeout=2000

####xop component properties####
xop.component.autoload=true
xop.component.autoload.poll.period.ms=5000
xop.component.autoload.dir=./plugins/

####Gump Properties####
Gump.Discovery=edu.drexel.xop.gump.sd.MultiInterfaceInstance
Gump.Multicast=mil.navy.nrl.protean.gump.transport.multicast.imp.JDKMulticastSocket
Gump.Transport=edu.drexel.xop.gump.NullTransportHandler
Gump.Proxy=edu.drexel.xop.gump.NullProxy

#Gump Norm options, only useful if Gump.Multicast is set to NormMulticast
Gump.Norm.CC=off
Gump.Norm.Sender.bufspace=5242880
Gump.Norm.Reciever.bufspace=5242880
Gump.Norm.SegmentSize=1400
Gump.Norm.BlockSize=64
Gump.Norm.NumParity=16
Gump.Norm.RepairWindowSize=1048576

\end{verbatim}
\caption{Sample XOP properties file}
\label{fig:sample-config-file}
\end{figure}

\subsection{Loading and Unloading Components}
Component loading is controlled via the \texttt{xop.component.autoload} configuration options. Components are jar files that contain a component.xml file that is used to configure them. 

\paragraph{Loading}
Placing the component jar file in the \texttt{xop.component.autoload.dir} directory with autoloading enabled will load the component into XOP
\paragraph{Unloading}
Removing the component jar file from the 
\texttt{xop.component.autoload.dir} directory will remove the component.

\section{Running XO}
There are two main ways to run XOP, dependent on how XOP was built.

Note that both these methods will launch a GUI application to determine the network interface to use if it is not specified.
\subsection{Release Bundled Version}
The bundled version, just binaries no source, has a run script (\texttt{bundle.sh}) that checks for Java and runs XOP. If no suitable Java is found it resorts to the bundled version. As a result of this the bundled version does not require Java to be installed to run (Linux only).

To Run:
\begin{verbatim}
$ ./bundle.sh
\end{verbatim}
\subsection{Built from Source Version}
If XOP was built from source the best way to run it is to execute \texttt{java -jar xop.jar} from the \texttt{dist/} directory.

To Run:
\begin{verbatim}
$ cd dist/
$ java -jar xop.jar
\end{verbatim}
\section{Notable Components}
\subsection{Multi User Chat (MUC)}
The main functionality of XOP is contained in the MUC plugin, which allows for transparent group communication on the MANET.
\subsubsection{Configuration}
The MUC component has a few configuration options that are outside the normal XOP configuration. 

\paragraph{\texttt{muc.port}}
Which port (for multicast) should the MUC component listen on.

Default: 5150
\paragraph{\texttt{muc.blocking}}
Determines whether the call to sendPacket blocks in the MUC component. Set to true, if you are getting an unusual latency (high).

Default: false
\paragraph{\texttt{muc.MultiInterface}}
Should MUC listen and write to all interfaces for multicast traffic.

Default: false
\paragraph{\texttt{muc.interfaces}}
Lists which interfaces MUC should listen on, comma delimited (overrides xop.device for MUC)

\textbf{Must be set if \texttt{muc.MultiInterface} is set to true. }

Default: undef
\subsection{XOP Gateway (XOG)}

\section{Known Issues}
\end{document}
