\sloppypar Service discovery is an important aspect of any automated distributed system and the more transient the distributed system's entities, the more problematic is this task.  In mobile ad-hoc networks (MANETs), where self-configuring infrastructureless networks of mobile devices are often connected by multi-hop wireless links, the challenge becomes significant with incongruous tradeoffs between state maintenance for reliability and bandwidth consumption for efficiency.  Potential solutions require techniques that can efficiently map the distributed application across a complexity of variable mobility patterns and underlying routing algorithms.  Existing standardized solutions for service discovery, such as mDNS, are inadequate because they employ a rather fixed set of timers that cannot adapt to different application scenarios and dynamics of the underlying network connectivity.   In this paper, we explore a combinatory approach that, on the one hand, leverages existing standardized service discovery messaging to ensure interoperability with mDNS and DNS, and then on the other,  provides flexible tuning of distributed modes of operation to serve the application's needs for a particular type of network. We discuss the application usages of such an approach and describe how these techniques can provide a bridging infrastructure for interoperability between mobile and fixed network infrastructures.  We show that by using this flexibility of tuning modes, we can create a proactive approach to service discovery, which we hypothesise can exhibit considerable efficiency and success rate gains over conventional techniques in transient mobile environments. We empirically prove this hypothesis by presenting results based on 96 different scenarios recorded using emulations representing real-world experiments that compare this approach with mDNS.      
