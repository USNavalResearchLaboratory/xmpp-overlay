\section{Conclusions}
\label{sec:conclusions}

In this \doctype, we have described the motivation, architecture, design, implementation and comparative analysis of INDI compared to an existing solutions for standardized service discovery on a local area network.   We have explained how  INDI achieves full interoperability with mDNS and that it offers attractive flexible service discovery profiles for optimization  in mobile networks. 

We have shown that INDI outperforms mDNS for a simple wireless networking mobile MANET scenario.  For bandwidth consumption, INDI achieved up to a six fold reduction in messages.  For success rates, even according to a modest quality of service metric for latencies of discovery and departure of services, INDI achieved at least a 17\% superior difference in results. It proved to be far more responsive and adaptable to network change.
 
Since the operating nature of INDI is peer-to-peer and therefore service providers advertise their own services, such gains make a significant difference. If INDI can provide more up-to-date service advertisements for exposing application endpoints, it follows therefore the connection to its endpoint to use the application is also likely to be more successful too.   This results in a far higher quality of experience for using services in a MANET environment in general.

Regarding future work we feel that INDI work could contribute in helping specify an open standard multi-hop multicast capable extension to mDNS and potentially other link local service discovery systems. Site-scoped multicast forwarding in mobile ad hoc networks is an example of where this could be applied in the short term.  Appropriate domains and rules need to be developed to make this both consistent and interoperable with mDNS and the DNS architecture.  Presently INDI provides a working prototype demonstrating that this type of design adaptation is achievable and beneficial.
