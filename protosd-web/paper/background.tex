INDI is designed to provide robust service discovery in the Mobile Ad-hoc network (MANET) environment \cite{basagni2004mobile, broch1998, perkins2000}, which has historically been far more focused on the development and testing of unicast routing protocols, and to a lesser extent, multicast routing and forwarding  \cite{lee1999demand,camp2002,macker2004simplified}. Even so, there are now many approaches to MANET routing and architectures and daily operational experience has been gained in their use e.g., operational community networks \cite{funkfeuer}.  However, far less research is focused on generic collaborative applications e.g. multiuser chat \cite{XMPPCore}, operating in such networks, and in particular, service discovery overlays to help automate group communications.

\sloppypar Highly scalable existing systems, such as some peer-to-peer systems (Gnutella, Limewire) and distributed hashtables (e.g. CAN \cite{can}, Chord \cite{chord}) for a WAN, tend to address transient connectivity through redundancy, providing multiple paths across the virtual overlay to provide resilience in the event of nodes leaving the network. Although such approaches works well for Internet applications, it has a large traffic overhead (through partial flooding) in terms of traffic and is not easily translated to a  environment that is more restricted in terms of bandwidth.  

For service discovery, there are number of approaches e.g, the aforementioned SSDP, mDNS, Bonjour and SLP, and many  have undergone an extensive standardization effort with a tight focus on the deployment environment that they address.  For the case of mDNS, it focuses on use with a site local LAN and provides interoperability with existing DNS servers for a WAN.  Therefore, the system is tuned specifically for this context with underlying timers being rigidly defined. Furthermore, mDNS also attempts to provide a single operational protocol to cover all use cases within a LAN and so it employs a rather generic combination of reactive and proactive service discovery schemes that work in tandem. This approach by definition requires more redundancy to provide the robustness and genericness requirements for a LAN, which inevitably leads to a higher overhead in communication.  

The extra message overhead associated with such an approach is generally unacceptable for a MANET, which typically has a far more restricted bandwidth. We will show the bandwidth connotations of this in Section \ref{sec:results} and we will also see that more specifically tuned deployments may well be more suited to the underlying mobility patterns of a network at a point in time e.g. employing more proactive-based schemes may suit more dynamic networks.  This is not a criticism of the of the mDNS approach at all because it designed specifically to address the needs of the LAN environment and to provide a fault tolerant and safe solution within that intended infrastructure.   One cannot design a single protocol to address all possible environments and in this \doctype, we argue this point strongly.  

In our work here therefore, we intentionally take a more hands off approach and instead provide a concise but functional set of parameters and profiles that can be used to apply the mDNS-style infrastructure in multiple ways depending on the needs of the application and the underlying mobility patterns.    We hope that this work might form a reasonable starting point for a taxonomy for addressing this space in future.      There is currently a complimentary early draft effort on-going within the IETF for Extended mDNS (xmDNS,  \cite{lynn2011}), which aims to  extend mDNS to site-local scope. Although this approach makes steps in defining a multicast address to accommodate site-scoped support of mDNS, it does not provide profiles and timing settings for such deployments and therefore we believe the work described in this \doctype~has a complimentary focus.

