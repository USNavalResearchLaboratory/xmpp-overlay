The emergence of Apple's Rendezvous for (now Bonjour) upon switching from SLP \cite{slp} with the launch of Mac OS X 10.2 ``Jaguar'' in August 2002, initiated a number of activities for standardization of a zero configuration protocol called Multicast Domain Name Service (mDNS, \cite{giordano2005dns,zeroconfbook}). Since then, mDNS has since gained probably the most widespread deployment across platforms of all of its predecessors.   Other systems, such as  the Service Location Protocol version 2 (SLPv2) \cite{guttman1999service} and the Simple Service Discovery Protocol (SSDP) in Microsoft's Universal Plug n' Play (UPnP) standard \cite{helal2002standards} are in use but have not gained quite as much impact.  

For wide area networks, service discovery has taken a somewhat simplified or application-specific  approach. Middleware systems such as  JINI \cite{jini} and Jxta \cite{Jxta} and SOA-based architectures such as UDDI directories \cite{uddi}, are based on simple mechanisms, supporting only basic multicast and/or unicast primitives for connecting to lookup servers, rendezvous nodes, and so forth. Most other service discovery in this space, even for the highly distributed peer-to-peer variety, tend to function at an application specific level once a node has connected to the application overlay using TCP, e.g. Limewire \cite{limewire} and Gnutella \cite{gnutella}.  The connection to the network itself e.g., the discovery of a ``Limewire peer'', is typically achieved through a centralized server list of good candidates. Although distributed query overlays (using rendezvous nodes) and distributed hash tables (DHTs, e.g. \cite{Ratnasamy01ascalable}) take different approaches, they also tend to be application specific by partitioning the network into sub-areas based on application-specific criteria and content e.g. file content, namespaces and cryptography keys, and so on.  


A similar approach through the use of a unicast connection is taken by mDNS to connect to the WAN, however, this has the added advantage that it leverages existing infrastructure for connecting a mDNS responder to a deployed DNS server. Since DNS is already ubiquitous on the Internet, programmers are already familiar with its interfaces and can deploy service advertisements using the same data structures that DNS does for other records.  In terms of uptake, dynamic discovery on a LAN and interoperability with existing infrastructure on a WAN therefore, it can be agued that mDNS seems to be gaining traction to become the potential de facto mechanism for standardized service discovery. 


For mobile ad-hoc networks (MANETs), research of service discovery has received less focus and there is a lack of extensive deployment experience in these environments. The bulk of past MANET work has mainly focused on improving the underlying transport protocols and routing algorithms rather than examining distributed service components, middleware, and other upper layers of the network system stack. There is a clear gap therefore between the MANET low-level routing protocol implementations facilitating packet delivery across highly transient multi-hop networks, and the somewhat high-level service discovery mechanisms, designed for far less transient devices on LANs.  None of the aforementioned service discovery approaches therefore address the MANET environment effectively.   Furthermore, the timer settings in existing protocols are typically statically defined through the standardization of the protocol but this restricts the effectiveness when heterogeneity and the persistence of services is more dynamic.  Present solutions therefore, by design, do not expose mechanisms or interfaces for fine timers or operating modes to provide flexibility and therefore widespread deployment of these protocols in a MANET is currently unsuitable. 

In this \doctype, we attempt to bridge these areas by, on the one hand, providing suitable service discovery mechanisms for addressing the dynamic nature of a MANET environment and yet, on the other hand, expose service messages in a format that is interoperable with the existing LAN and WAN infrastructure provided by mDNS.   To accomplish this task we integrate our previous Independent Network Discovery System (INDI) algorithms \cite{Macker2010} into JmDNS via its timer interfaces for adapting functionality for the MANET environment. By leveraging existing mDNS messaging and communication, INDI remains interoperable across operating systems and applications e.g., Apple and Windows Bonjour\footnote{http://www.apple.com/support/bonjour/} and Linux-based AVAHI\footnote{http://avahi.org/}. Therefore, it can ease both the transition burden and the interoperation with fixed-infrastructure networks by providing a standardized means to interact with global service-based infrastructure.    This reduces the problem of interoperability and the need to provide a single point translation from the MANET to the enterprise by creating a homogeneous messaging infrastructure. All that is needed thereafter in order to bridge the networks in terms of service discovery is to provide simple lightweight repeaters to relay the messages between them.  

In this work therefore, we attempt to address a number of research objectives, such as:

\begin{itemize}

\item Is it possible to comply with the messaging infrastructure of standardized infrastructure mDNS and DNS, to provide interoperability with the LAN and WAN environments and still meet the needs of a MANET? 
\item How can these diverse environments be combined to facilitate one service discovery layer that can bridge between these different infrastructures?
\item What are the considerations for deploying a service infrastructure in a MANET?
\item What are the tradeoffs for employing a more proactive announcement scheme and shorter time to live settings for addressing the dynamic connectivity of nodes in a MANET ? 
\item Does the mDNS service discovery approach perform more efficiently than a proactive scheme for a MANET environment? 
\item Does the mDNS service discovery approach have a lower overhead than using a more proactive scheme for a  MANET environment? 
\end{itemize}

We hypothesize that by layering decentralized and mobility tolerant service discovery solutions upon an existing mDNS software infrastructure, we can meet these constraints. We also hypothesize that such a hybrid system would significantly outperform the existing service discovery algorithms and timers employed within mDNS for use within a dynamic context.  This \doctype~therefore describes the  research conducted in this space, building from previous results from MANET experiments, extracting requirements, and architecting, designing and integrating these techniques within an existing mDNS infrastructure.  

